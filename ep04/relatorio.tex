\documentclass[12pt, a4paper]{article}

\usepackage[portuguese]{babel}
\usepackage[utf8]{inputenc}
\usepackage{xspace}
\usepackage{amsmath}
\usepackage{nccmath}
\usepackage{mathtools}
\usepackage{bbm}
\usepackage{multirow}
\usepackage{amsfonts}
\usepackage{amssymb}
\usepackage[margin=1.0in]{geometry}
\usepackage[decimalsymbol=comma]{siunitx}
\setlength{\parindent}{0pt}
\usepackage{tikz}
\usepackage{listings}
\usetikzlibrary{arrows,shapes,automata,petri,positioning,calc}

\title{MAC0422 - Sistemas Operacionais}
\author{Matheus Santos - 10297672 \\ Vitor Barbosa Sério - 7627627}
\date{\today}

\newcommand\eqdef{\stackrel{\mathclap{\normalfont\mbox{def}}}{=}}
\newcommand\eqdub{\stackrel{\mathclap{\normalfont\mbox{?}}}{=}}
\newcommand\leqdub{\stackrel{\mathclap{\normalfont\mbox{?}}}{\leq}}
\newcommand\geqdub{\stackrel{\mathclap{\normalfont\mbox{?}}}{\geq}}
\DeclarePairedDelimiter{\ceil}{\lceil}{\rceil}
\DeclarePairedDelimiter\floor{\lfloor}{\rfloor}

\newcommand\Oh[1]{\text{O}(#1)}
\newcommand\Thetah[1]{\Theta(#1)}

\newcommand{\benum}{\begin{enumerate}}
\newcommand{\eenum}{\end{enumerate}}

\begin{document}


\makeatletter
\begin{flushright}
\@date
\end{flushright}
\begin{center}
{\Large \@title \\}
{\large \@author}
\end{center}
\makeatother

Esse trabalho implementa a chamada de sistema \texttt{open{\_}tmp(path, modo)}, que abre e cria arquivos temporários, sendo \texttt{path} o caminho absoluto ou relativo para a criação do arquivo e \texttt{modo} o modo de alteração do mesmo. Os modos de alteração seguem o mesmo padrão da função \texttt{fopen} (`r', `w', `a', `r+', `w+ e `a+').

Abaixo segue a lista dos arquivos alterados ou criados e qual papel eles desempenham na implementação:

\begin{itemize}

\item \begin{itemize}

\item Diretório: \texttt{/usr/src/servers/fs/}

\item Arquivos: \texttt{proto.h, table.c, open.c, read.c, misc.c}

\item Função: Implementa a função e o protótipo da mesma para a chamada de sistema no nível do \textit{File System} (FS) através da função \texttt{do{\_}open{\_}tmp}. Além disso, foram alteradas as seguintes funções: \texttt{do{\_}slink}, para impedir a craição de links simbólicos que apontem para arquivos temporários; \texttt{read{\_}write}, para permitir a leitura e escrita em arquivos temporários; e \texttt{free{\_}proc}, para remover os arquivos temporários quando os processos que apontavam para os mesmos se encerram.

\end{itemize}

\item \begin{itemize}

\item Diretório: \texttt{/usr/src/lib/posix/}

\item Arquivos: \texttt{{\_}open{\_}tmp.c, Makefile.in}

\item Função: Implementa a função para usuário, \texttt{open{\_}tmp}, que recebe \texttt{path} e \texttt{modo} e faz a chamada para o FS.

\end{itemize}

\item \begin{itemize}

\item Diretório: \texttt{/usr/src/include/}

\item Arquivos: \texttt{fcntl.h, minix/callnr.h, minix/const.h, sys/stat.h}

\item Função: Define o protótipo da função \texttt{open{\_}tmp}, além dos macros \texttt{OPENTMP} e \texttt{I{\_}TEMPORARY} para o número da chamada de sistema referente a \texttt{do{\_}open{\_}tmp} e para o novo tipo de arquivo, respectivamente. Também foram definidos os macros \texttt{S{\_}IFTMP} e \texttt{S{\_}ISTMP(m)} para verificar se um arquivo é temporário através do struct \texttt{stat}.

\end{itemize}

\end{itemize}

\end{document}