\documentclass[12pt, a4paper]{article}

\usepackage[portuguese]{babel}
\usepackage[utf8]{inputenc}
\usepackage{xspace}
\usepackage{amsmath}
\usepackage{nccmath}
\usepackage{mathtools}
\usepackage{bbm}
\usepackage{multirow}
\usepackage{amsfonts}
\usepackage{amssymb}
\usepackage[margin=1.0in]{geometry}
\usepackage[decimalsymbol=comma]{siunitx}
\setlength{\parindent}{0pt}
\usepackage{tikz}
\usepackage{listings}
\usetikzlibrary{arrows,shapes,automata,petri,positioning,calc}

\title{MAC0422 - Sistemas Operacionais}
\author{Matheus Santos - 10297672 \\ Vitor Barbosa Sério - 7627627}
\date{\today}

\newcommand\eqdef{\stackrel{\mathclap{\normalfont\mbox{def}}}{=}}
\newcommand\eqdub{\stackrel{\mathclap{\normalfont\mbox{?}}}{=}}
\newcommand\leqdub{\stackrel{\mathclap{\normalfont\mbox{?}}}{\leq}}
\newcommand\geqdub{\stackrel{\mathclap{\normalfont\mbox{?}}}{\geq}}
\DeclarePairedDelimiter{\ceil}{\lceil}{\rceil}
\DeclarePairedDelimiter\floor{\lfloor}{\rfloor}

\newcommand\Oh[1]{\text{O}(#1)}
\newcommand\Thetah[1]{\Theta(#1)}

\newcommand{\benum}{\begin{enumerate}}
\newcommand{\eenum}{\end{enumerate}}

\begin{document}


\makeatletter
\begin{flushright}
\@date
\end{flushright}
\begin{center}
{\Large \@title \\}
{\large \@author}
\end{center}
\makeatother

Esse trabalho implementa a chamada de sistema \texttt{chpriority(pid, priority)}, que altera a prioridade de um processo filho (o que possui o PID passado) para \texttt{priority}. A chamada retorna \texttt{priority} se a alteração foi feita com sucesso, -1 se \texttt{pid} não for de um processo filho e -2 se \texttt{priority} não for um valor permitido pelo sistema.

Abaixo segue a lista dos arquivos alterados ou criados e qual papel eles desempenham na implementação:

\begin{itemize}

\item \begin{itemize}

\item Diretório: \texttt{/usr/src/servers/pm/}

\item Arquivos: \texttt{proto.h, table.c, chpriority.c, Makefile}

\item Função: Implementa a chamada de sistema no nível do \textit{Process Manager} (PM) e manda a chamada pro \textit{kernel} alterar a prioridade do processo.

\end{itemize}

\item \begin{itemize}

\item Diretório: \texttt{/usr/src/servers/is/}

\item Arquivos: \texttt{dmp{\_}kernel.c, dmp.c, proto.h}

\item Função: Altera o funcionamento da tecla \texttt{F4}, para mostrar a lista de processos, com suas prioridades, PIDs, tempos de CPU e de sistema e endereço do ponteiro na pilha.

\end{itemize}

\item \begin{itemize}

\item Diretório: \texttt{/usr/src/lib/posix/}

\item Arquivos: \texttt{{\_}chpriority.c, Makefile.in}

\item Função: Implementa a função de nível de usuário, que faz a chamada para o PM.

\end{itemize}

\item \begin{itemize}

\item Diretório: \texttt{/usr/src/lib/syslib/}

\item Arquivos: \texttt{sys{\_}chpriority.c}

\item Função: Implementa a função que permite que o PM chame o \textit{system task}, que então faz a chamada para o \textit{kernel}.

\end{itemize}

\item \begin{itemize}

\item Diretório: \texttt{/usr/src/include/}

\item Arquivos: \texttt{unistd.h, minix/callnr.h, minix/com.h}

\item Função: Define os macros \texttt{CHPRIORITY} e \texttt{SYS{\_}CHPRIORITY} para os números da chamada de sistema no nível do PM e do \textit{kernel}, respectivamente e também o protótipo de \texttt{chpriority} para o usuário.

\end{itemize}

\item \begin{itemize}

\item Diretório: \texttt{/usr/src/kernel/}

\item Arquivos: \texttt{system.h, system.c, system/do{\_}chpriority.c}

\item Função: Implementa a chamada no nível do \testit{kernel}.
\end{itemize}

\end{itemize}


\textbf{IMPORTANTE:} Para iniciar a máquinha virtual pode ser necessário usar o comando `\texttt{boot d0p0}'.

\end{document}