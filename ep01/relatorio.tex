\documentclass[12pt, a4paper]{article}

\usepackage[portuguese]{babel}
\usepackage[utf8]{inputenc}
\usepackage{xspace}
\usepackage{amsmath}
\usepackage{nccmath}
\usepackage{mathtools}
\usepackage{bbm}
\usepackage{multirow}
\usepackage{amsfonts}
\usepackage{amssymb}
\usepackage[margin=1.0in]{geometry}
\usepackage[decimalsymbol=comma]{siunitx}
\setlength{\parindent}{0pt}
\usepackage{tikz}
\usepackage{listings}
\usetikzlibrary{arrows,shapes,automata,petri,positioning,calc}

\title{MAC0422 - Sistemas Operacionais}
\author{Matheus Santos - 10297672 \\ Vitor Barbosa Sério - 7627627}
\date{\today}

\newcommand\eqdef{\stackrel{\mathclap{\normalfont\mbox{def}}}{=}}
\newcommand\eqdub{\stackrel{\mathclap{\normalfont\mbox{?}}}{=}}
\newcommand\leqdub{\stackrel{\mathclap{\normalfont\mbox{?}}}{\leq}}
\newcommand\geqdub{\stackrel{\mathclap{\normalfont\mbox{?}}}{\geq}}
\DeclarePairedDelimiter{\ceil}{\lceil}{\rceil}
\DeclarePairedDelimiter\floor{\lfloor}{\rfloor}

\newcommand\Oh[1]{\text{O}(#1)}
\newcommand\Thetah[1]{\Theta(#1)}

\newcommand{\benum}{\begin{enumerate}}
\newcommand{\eenum}{\end{enumerate}}

\begin{document}


\makeatletter
\begin{flushright}
\@date
\end{flushright}
\begin{center}
{\Large \@title \\}
{\large \@author}
\end{center}
\makeatother

Esse trabalho simula uma shell capaz de interpretar 5 comandos distindos: \texttt{protegepracaramba}, \texttt{liberageral}, \texttt{rodeveja}, \texttt{rode} e \texttt{quit}, que funcionam das seguintes formas:

\begin{itemize}
\item \texttt{protegepracaramba <arquivo>}: coloca proteção 000 em \texttt{arquivo}, utilizando as chamadas \texttt{fork}, que cria um processo filho, \texttt{execve}, para executar outro programa no lugar do processo filho, e \texttt{chmod}, que é a chamada executada por \texttt{execve} e que altera as permissões de \texttt{arquivo}.


\item \texttt{liberageral <arquivo>}: coloca proteção 777 em \texttt{arquivo}, utilizando as chamadas \texttt{fork}, que cria um processo filho, \texttt{execve}, para executar outro programa no lugar do processo filho, e \texttt{chmod}, que é a chamada executada por \texttt{execve} e que altera as permissões de \texttt{arquivo}.


\item \texttt{rodeveja <programa>}: executa \texttt{programa} e imprime o valor de retorno do mesmo ao final, utilizando as chamadas \texttt{fork}, que cria um processo filho, \texttt{execve}, para executar \texttt{programa} no lugar do processo filho, e \texttt{wait}, para que o processo pai espere o processo filho terminar. Dessa forma, como o processo filho está executando em primeiro plano (foreground) a shell não consegue executar outro comando enquanto o processo filho não terminar (ctrl+c encerra o processo filho).
\item \texttt{rode <programa>}: executa \texttt{programa}, sem imprimir o valor de retorno e também monopoliza o teclado e o mouse, ou seja, o processo filho estará executando em segundo plano (background) e permitindo que a shell execute outros comandos. Isso é feito utilizando as chamadas \texttt{fork}, que cria um processo filho, e \texttt{execve}, para executar \texttt{programa} no lugar do processo filho criado (ctrl+c não encerra o processo filho).
\item \texttt{quit}: encerra a shell.
\end{itemize}

\end{document}